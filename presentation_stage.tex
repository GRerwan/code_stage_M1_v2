\documentclass{beamer}

% Utilisation du thème Darmstadt
\usetheme{Darmstadt}

% Packages nécessaires pour la couleur
\usepackage{xcolor}
\usepackage{graphicx}

% Définition des couleurs personnalisées pour les sections
\definecolor{mysectioncolor}{RGB}{255, 255, 255} % Couleur pour la section courante
\definecolor{myothersectionscolor}{RGB}{255, 255, 255} % Couleur pour les autres sections BLANC

\definecolor{blanc}{RGB}{255, 255, 255}
\definecolor{bleuNRJLAB}{RGB}{20,67,100}
\definecolor{bleuNRJLABfonce}{RGB}{20,67,100}
\definecolor{orangeNRJLAB}{RGB}{245,155,27}


\usecolortheme[named=bleuNRJLAB]{structure}
% Configuration du pied de page pour afficher le numéro de page à droite
%\setbeamertemplate{footline}[frame number]

% Configuration du pied de page
\setbeamertemplate{footline}{
	\begin{beamercolorbox}[wd=\paperwidth,ht=2.25ex,dp=1ex,leftskip=1em,rightskip=1em]{author in head/foot}
		\usebeamerfont{author in head/foot}\insertshortauthor\hfill\insertshortinstitute\hfill\insertframenumber{}/\inserttotalframenumber
	\end{beamercolorbox}
}

\newenvironment{blockorange}[1]{%
	\setbeamercolor{block title example}{bg=orangeNRJLAB,fg=white}%
	\setbeamercolor{block body example}{bg=orangeNRJLAB!10,fg=black}%
	\begin{exampleblock}{#1}}{\end{exampleblock}}

\newenvironment{blockbleu}[1]{%
	\setbeamercolor{block title example}{bg=bleuNRJLAB,fg=white}%
	\setbeamercolor{block body example}{bg=bleuNRJLAB!10,fg=black}%
	\begin{exampleblock}{#1}}{\end{exampleblock}}

\setbeamercolor{itemize item}{fg=black}


% Définir les marges de gauche et de droite
\setbeamersize{text margin left=0.5cm, text margin right=0.5cm}

% Modifier la taille de la police pour le headline
\setbeamerfont{subsection in head/foot}{size=\normalsize }


% Informations sur le document
\title{Inter-comparaison et validation entre les messures in situe et les estimations satellite de l'irradiance solaire incident dans la zone du Sud Ouest de l'Océan Indien}
\author{GRONDIN Erwan}
\institute{Universié de La Réunion}
\date{\today}

\begin{document}
	

	
\begin{frame}
\titlepage

\centering
\includegraphics[width=0.5\linewidth]{C:/Users/erwan/Desktop/UNIVERSITE/LOGO/energy_labb}
\includegraphics[width=0.5\linewidth]{C:/Users/erwan/Desktop/UNIVERSITE/LOGO/energy_labb}

\thispagestyle{empty} % Pour ne pas afficher l'en-tête sur cette page
\end{frame}

\begin{frame}
	\frametitle{Sommaire}
	\tableofcontents[]
	\thispagestyle{empty}
\end{frame}

\section{Introduction}
\begin{frame}
	\frametitle{Introduction}
	
\end{frame}

\section{Données}
\subsection{Données de l'étude}
\begin{frame}
	\frametitle{Données}
	\begin{minipage}[t]{0.48\textwidth}
		\begin{itemize}
			\item Irradiance Horizontale Globale (GHI) en $W/m^2$.
			\item Irradiance Normale Directe (DNI) en $W/m^2$ .
		\end{itemize}
	\end{minipage}
	\hfill
	\begin{minipage}[t]{0.48\textwidth}
		\begin{figure}
			\centering
			\includegraphics[width=0.7\linewidth]{img/lien_entre_ghi_dni_dhi}
		\end{figure}
	\end{minipage}
	\vfill
	\begin{minipage}[t]{0.48\textwidth}
		\begin{figure}
			\centering
			\includegraphics[width=0.7\linewidth]{img/four_solaire}
			\caption{Système à concentration solaire (DNI)}
			\label{fig:foursolaire}
		\end{figure}
		
	\end{minipage}
	\hfill
	\begin{minipage}[t]{0.48\textwidth}
		\begin{figure}
			\centering
			\includegraphics[width=0.7\linewidth]{img/panneau_solaire}
			\caption{Panneau solaire (GHI)}
			\label{fig:Panneausolaire}
		\end{figure}
	\end{minipage}
\end{frame}


\subsection{Les mesures in situ IOS-net}
\begin{frame}
	
	% Première moitié de la page en hauteur, et toute la largeur de la frame
	\begin{minipage}[t][0.2\textheight][t]{\textwidth}
		\begin{blockorange}{Appareils de mesures}
			\small
			\begin{itemize}
				\setlength{\itemsep}{0.5pt} % Modification de l'espacement entre les items
				\item \textbf{SPN1 / CMP22} : Pour les données de GHI et DHI
				\item \textbf{CHP1} : Pour les données de DNI (BSRN)
			\end{itemize}
		\end{blockorange}
	\end{minipage}
	
	\vfill
	
	% Deuxième moitié de la page, divisée en deux colonnes
	\begin{minipage}[c][0.8\textheight][c]{0.45\textwidth}
		\begin{blockbleu}{\'Equipement des stations}
			\small
			\begin{itemize}
				\setlength{\itemsep}{0.5pt} % Modification de l'espacement entre les items
				\item Pyranomètre SPN1
				\item Transmetteur météorologique WXT530
				\item Radiomètre UV
				\item Centrale d’acquisition
				\item \'Eolienne
				\item Panneau photovoltaïque
				\item Régulateur de charge
				\item Batterie de plomb 20 AH
			\end{itemize}
		\end{blockbleu}
	\end{minipage}
	\hfill
	\begin{minipage}[c][0.8\textheight][c]{0.45\textwidth}
		\includegraphics[trim=3cm 0cm 3cm 1cm, clip, width=1\linewidth]{img/SWIO_station}
	\end{minipage}
	
\end{frame}



\subsection{Les estimations SARAH-3}



\begin{frame}
	\begin{minipage}[c][\textheight][c]{0.45\textwidth}
		\includegraphics[width=1.2\linewidth]{img/exemple_cm_saf_1}
	\end{minipage}
	\hfill
	\begin{minipage}[c][\textheight][c]{0.45\textwidth}
		\begin{blockorange}{Caractéristiques des données SARAH-3}
			\small
			\begin{itemize}
				\setlength{\itemsep}{0.5pt} % Modification de l'espacement entre les items
				\item\textbf{Spatial grid } : ±65° de longitude / ±65° de latitude
				\item \textbf{Spatial precision } : 0,05° x 0,05°
				\item \textbf{Temporal grid} : 01/01/1983 to 01/04/2024
				\item \textbf{Product} : CAL, DAL, DNI, PAR , SDU, SID and SIS
			\end{itemize}
		\end{blockorange}
	\end{minipage}
\end{frame}



\section{Méthodologie}
\subsection{Importation des donnée IOS-net}
\begin{frame}
	%\frametitle{Données}
	Contenu des travaux connexes.
\end{frame}
\subsection{Estimation du DNI (IOS-net)}
\begin{frame}
	%\frametitle{Données}
	Contenu des travaux connexes.
\end{frame}


\subsection{Contrôle de qualité}
\begin{frame}
	%\frametitle{Données}
	Contenu des travaux connexes.
\end{frame}
\subsection{Moyenne temporelle}
\begin{frame}
	%\frametitle{Données}
	Contenu des travaux connexes.
\end{frame}
\subsection{Importation des données SARAH-3}
\begin{frame}
	%\frametitle{Données}
	Contenu des travaux connexes.
\end{frame}
\subsection{Comparaison des méthodes de moyennes}
\begin{frame}
	%\frametitle{Données}
	Contenu des travaux connexes.
\end{frame}
\subsection{Adaptation de données géographique}
\begin{frame}
	%\frametitle{Données}
	Contenu des travaux connexes.
\end{frame}
\subsection{Importation des donnée IOS-net}
\begin{frame}
	\frametitle{Méthodologie}
	Contenu de la méthodologie.
\end{frame}

\section{Résultats}
\subsection{Comparaison DNI}
\begin{frame}
	%\frametitle{Données}
	Contenu des travaux connexes.
\end{frame}
\subsection{Comparaison GHI}
\begin{frame}
	%\frametitle{Données}
	Contenu des travaux connexes.
\end{frame}
\subsection{Indicateur de précision}
\begin{frame}
	\frametitle{Résultats}
	Contenu des résultats.
\end{frame}

\section{Conclusion et Discussion}
\begin{frame}
	\frametitle{Conclusion et Discussion}
	Contenu de la conclusion.
\end{frame}
	
\end{document}
