\documentclass[8pt]{beamer}


% Utilisation du thème Darmstadt
\usetheme{Darmstadt}

% Packages nécessaires pour la couleur
\usepackage{xcolor}
\usepackage{siunitx}
\usepackage{amsmath,amssymb,amsfonts}
\usepackage{listings}
\usepackage{graphicx}
\usepackage{textcomp}
\usepackage{url}
\usepackage{comment}
\def\UrlFont{\rmfamily}
\usepackage{booktabs}
\usepackage{multicol}
\usepackage{float} 
\usepackage{multirow, array}
\usepackage[utf8]{inputenc} 
\usepackage[style=numeric,backend=bibtex,sorting=none,doi=true]{biblatex}
\addbibresource{biblio}
\usepackage{academicons}
\usepackage{svg}
\usepackage{orcidlink}
\usepackage{afterpage}
\usepackage{longtable}
\usepackage[table]{xcolor}
\usepackage{colortbl}
\usepackage{subcaption} % Package pour les sous-figures
\usepackage{hyperref}
\usepackage{cleveref}


% Définition des couleurs personnalisées pour les sections
\definecolor{mysectioncolor}{RGB}{255, 255, 255} % Couleur pour la section courante
\definecolor{myothersectionscolor}{RGB}{255, 255, 255} % Couleur pour les autres sections BLANC

\definecolor{blanc}{RGB}{255, 255, 255}
\definecolor{bleuNRJLAB}{RGB}{20,67,100}
\definecolor{bleuNRJLABfonce}{RGB}{20,67,100}
\definecolor{orangeNRJLAB}{RGB}{245,155,27}


\usecolortheme[named=bleuNRJLAB]{structure}
% Configuration du pied de page pour afficher le numéro de page à droite
%\setbeamertemplate{footline}[frame number]

% Configuration du pied de page
\setbeamertemplate{footline}{
	\begin{beamercolorbox}[wd=\paperwidth,ht=2.25ex,dp=1ex,leftskip=1em,rightskip=1em]{author in head/foot}
		\usebeamerfont{author in head/foot}\insertshortauthor\hfill\insertshortinstitute\hfill\insertframenumber{}/\inserttotalframenumber
	\end{beamercolorbox}
}

\newenvironment{blockorange}[1]{%
	\setbeamercolor{block title example}{bg=orangeNRJLAB,fg=white}%
	\setbeamercolor{block body example}{bg=orangeNRJLAB!10,fg=black}%
	\begin{exampleblock}{#1}}{\end{exampleblock}}

\newenvironment{blockbleu}[1]{%
	\setbeamercolor{block title example}{bg=bleuNRJLAB,fg=white}%
	\setbeamercolor{block body example}{bg=bleuNRJLAB!10,fg=black}%
	\begin{exampleblock}{#1}}{\end{exampleblock}}

\setbeamercolor{itemize item}{fg=black}


% Définir les marges de gauche et de droite
\setbeamersize{text margin left=0.5cm, text margin right=0.5cm}

% Modifier la taille de la police pour le headline
%\setbeamerfont{subsection in head/foot}{size=\small }


% Informations sur le document
\title{Inter-comparaison et validation entre les messures in situe et les estimations satellite de l'irradiance solaire incident dans la zone du Sud Ouest de l'Océan Indien}
\author{GRONDIN Erwan}
\institute{Universié de La Réunion}
\titlegraphic{\includegraphics[width=0.4\textheight]{img/logo_UFRST}\quad\quad\quad\includegraphics[width=0.4\textheight]{img/energy_labb}}
\date{\today}

\begin{document}
	

	
\begin{frame}
\textbf{STAGE M1}: 2 mois\\
\textbf{Encadrants} : Dr.Béatrice MOREL et Dr.Dominique GRONDIN
\titlepage
\thispagestyle{empty} % Pour ne pas afficher l'en-tête sur cette page
\end{frame}

\begin{frame}
	\frametitle{Sommaire}
	\tableofcontents[]
	\thispagestyle{empty}
\end{frame}

\section{Introduction}
\begin{frame}
	\frametitle{Introduction}
	\includegraphics[width=0.7\linewidth]{img/logo_osu_r}
	\includegraphics[width=0.7\linewidth]{img/logo_osu_r1}
\end{frame}

\section{Données}
\subsection{Données de l'étude}
\begin{frame}
	\frametitle{Données}
	\begin{minipage}[t]{0.48\textwidth}
		\begin{itemize}
			\item Irradiance Horizontale Globale (GHI) en $W/m^2$.
			\item Irradiance Normale Directe (DNI) en $W/m^2$ .
		\end{itemize}
	\end{minipage}
	\hfill
	\begin{minipage}[t]{0.48\textwidth}
		\begin{figure}
			\centering
			\includegraphics[width=0.7\linewidth]{img/lien_entre_ghi_dni_dhi}
		\end{figure}
	\end{minipage}
	\vfill
	\begin{minipage}[t]{0.48\textwidth}
		\begin{figure}
			\centering
			\includegraphics[width=0.7\linewidth]{img/four_solaire}
			\caption{Système à concentration solaire (DNI)}
			\label{fig:foursolaire}
		\end{figure}
		
	\end{minipage}
	\hfill
	\begin{minipage}[t]{0.48\textwidth}
		\begin{figure}
			\centering
			\includegraphics[width=0.7\linewidth]{img/panneau_solaire}
			\caption{Panneau solaire (GHI)}
			\label{fig:Panneausolaire}
		\end{figure}
	\end{minipage}
\end{frame}


\subsection{Les mesures in situ IOS-net}
\begin{frame}
	
	% Première moitié de la page en hauteur, et toute la largeur de la frame
	\begin{minipage}[t][0.2\textheight][t]{\textwidth}
		\begin{blockorange}{Appareils de mesures}
			\small
			\begin{itemize}
				\setlength{\itemsep}{0.5pt} % Modification de l'espacement entre les items
				\item \textbf{SPN1 / CMP22} : Pour les données de GHI et DHI
				\item \textbf{CHP1} : Pour les données de DNI (BSRN)
			\end{itemize}
		\end{blockorange}
	\end{minipage}
	
	\vfill
	
	% Deuxième moitié de la page, divisée en deux colonnes
	\begin{minipage}[c][0.8\textheight][c]{0.45\textwidth}
		\begin{blockbleu}{\'Equipement des stations}
			\small
			\begin{itemize}
				\setlength{\itemsep}{0.5pt} % Modification de l'espacement entre les items
				\item Pyranomètre SPN1
				\item Transmetteur météorologique WXT530
				\item Radiomètre UV
				\item Centrale d’acquisition
				\item \'Eolienne
				\item Panneau photovoltaïque
				\item Régulateur de charge
				\item Batterie de plomb 20 AH
			\end{itemize}
		\end{blockbleu}
	\end{minipage}
	\hfill
	\begin{minipage}[c][0.8\textheight][c]{0.45\textwidth}
		\includegraphics[trim=3cm 0cm 3cm 1cm, clip, width=1\linewidth]{img/SWIO_station}
	\end{minipage}
	
\end{frame}



\subsection{Les estimations SARAH-3}



\begin{frame}
	\begin{minipage}[c][\textheight][c]{0.45\textwidth}
		\includegraphics[width=1.2\linewidth]{img/exemple_cm_saf_1}
	\end{minipage}
	\hfill
	\begin{minipage}[c][\textheight][c]{0.45\textwidth}
		\begin{blockorange}{Caractéristiques des données SARAH-3}
			%\small
			%\setlength{\itemsep}{0.5pt} % Modification de l'espacement entre les items
			\textbullet \ \textbf{Spatial grid } : ±65° de longitude / ±65° de latitude\\
			\textbullet \ \textbf{Spatial precision } : 0,05° x 0,05°\\
			\textbullet \ \textbf{Temporal grid} : 01/01/1983 to 01/04/2024\\
			\textbullet \ \textbf{Product} : CAL, DAL, DNI, PAR , SDU, SID and SIS
		\end{blockorange}
	\end{minipage}
\end{frame}



\section{Méthodologie}
\subsection{IOS-net data}
\begin{frame}
	\frametitle{Importation des donnée IOS-net}
	\begin{minipage}[c][\textheight][t]{0.35\textwidth}
		\begin{blockorange}{Code python}
			\textbullet \ Récupérer tous les liens \textbf{netCDF} avec requêtes \textit{HTTP} et \textit{BeautifulSoup}
			
			\textbullet \ Sauvegarder les liens une \textbf{liste} structurée
			
			\textbullet \ Conversion de tous les fichiers netCDF en \textbf{Dataframe} avec \textit{xarray}.
			
			\textbullet \ \textbf{Sélection} des données utiles (GHI,DHI,DNI et timestamp)

		\end{blockorange}
	\end{minipage}
	\hfill
	\begin{minipage}[c][\textheight][t]{0.6\textwidth}
		\begin{minipage}[t][0.3\textheight][t]{\textwidth}
			\includegraphics[width=\linewidth]{img/ios_net_TDS}
		\end{minipage}
		\vfill
		\begin{minipage}[t][0.69\textheight][t]{\textwidth}
			\includegraphics[width=\linewidth]{img/all_name_var}
		\end{minipage}
	\end{minipage}
\end{frame}


\begin{frame}
	\frametitle{Estimation du DNI}
	\begin{minipage}[c][0.4\textheight][t]{0.55\textwidth}
		\begin{blockbleu}{\'Estimation du DNI ($W/m^2$)}
			\begin{equation}
				DNI = \frac{1}{\mu_0}(GHI - DHI)
			\end{equation}
			
			Avec $\mu_0$ le cosinus du l'angle zenithale
			
		\end{blockbleu}
	\end{minipage}
	\hfill
	\begin{minipage}[c][0.4\textheight][t]{0.35\textwidth}
		\begin{blockorange}{Considération de Pvlib}
			\textbullet \ Latitude
			
			\textbullet \ Longitude
			
			\textbullet \ Altitude
			
			\textbullet \ Time Zone
			
			\textbullet \ Timestamp
			
		\end{blockorange}
	\end{minipage}
	
	\begin{figure}
		\centering
		\includegraphics[width=1\linewidth]{img/df_aure_example}
		\caption{Estimation du DNI pour la stations "aurere" à La Réunion}
		\label{fig:dfaureexample}
	\end{figure}
	
\end{frame}


\begin{frame}
	\frametitle{Contrôle de qualité}
	\small % Reduce font size in this slide
	\vspace{0.1cm}
	\begin{center}
		\textbf{Limites physiques} avec $S_a$ la radiation solaire extraterrestre
	\end{center}
	
	\begin{columns}[T] % [T] ensures correct vertical alignment
		\begin{column}{0.3\linewidth} % Left column
			\includegraphics[width=\linewidth]{img/flow_data/ghi/amitie}\\[1 pt]
			\begin{center}
				\textbf{QC1- GHI-($W/m^2$)}
			\end{center}
			\begin{equation}
				S_a \times 1.5 \times \mu_0^{1.2} + 100
			\end{equation}
		\end{column}
		\begin{column}{0.3\linewidth} % Center column
			\includegraphics[width=\linewidth]{img/flow_data/dhi/amitie}\\[1pt]
			\begin{center}
				\textbf{QC1- DHI-($W/m^2$)}
			\end{center}
			\begin{equation}
				S_a \times 0.95 \times \mu_0^{1.2} + 50
			\end{equation}
		\end{column}
		\begin{column}{0.3\linewidth} % Right column
			\includegraphics[width=\linewidth]{img/flow_data/dni/amitie}\\[1pt]
			\begin{center}
				\textbf{QC1- DNI-($W/m^2$)}
			\end{center}
			\begin{equation}
				S_a
			\end{equation}
		\end{column}
	\end{columns}
\end{frame}

\begin{frame}
	\frametitle{Moyenne temporelle}
	
	\begin{minipage}[c][\textheight][t]{0.58\textwidth}
		\centering
		\includegraphics[height=0.85\textheight]{img/comparison_mean_method_at_BSRNof_CHP1_dni_ground_URBSRN}
	\end{minipage}
	\hfill
	\begin{minipage}[c][\textheight][t]{0.37\textwidth}
		\centering
		\includegraphics[height=0.85\textheight]{img/mean_fonction}
	\end{minipage}
\end{frame}

\subsection{SARAH-3 data}
\begin{frame}
	\frametitle{Importation des données SARAH-3}
	
	\begin{minipage}[c][\textheight][t]{0.48\textwidth}
		\vspace{1cm}
		\begin{minipage}[c][0.3\textheight][t]{\textwidth}
			
			\begin{table}
				\centering
				\renewcommand{\arraystretch}{1} % Ajustement de la hauteur des lignes
				\footnotesize
				\small
				\rowcolors{1}{gray!25}{white}
				\begin{tabular}{>{\fontsize{5}{6}\selectfont\bfseries}l >{\fontsize{5}{6}\selectfont}c >{\fontsize{5}{6}\selectfont}c}
					\toprule
					\textbf{ID}  &\textbf{GHI} &\textbf{DNI}\\
					\midrule
					Product group & Climate Data Records & Climate Data Records \\
					Product family & SARAH ed. 3.0 & SARAH ed. 3.0 \\
					Product name & SIS & DNI \\
					Temporal resolution & Instantaneous & Instantaneous \\
					
					\bottomrule
				\end{tabular}
				\label{param_upload}
			\end{table}
		\end{minipage}
		\vfill
		\begin{minipage}[c][0.65\textheight][t]{\textwidth}
			\begin{figure}[t]
				\centering
				\footnotesize
				\includegraphics[width=\textwidth]{img/diagrame_code}
				\caption{Diagram of the interface of gnu-MAGIC / SPECMAGIC to the atmospheric input and the satellite observations. (Source : \href{https://www.cmsaf.eu/SharedDocs/Literatur/document/2023/saf_cm_dwd_atbd_sarah_3_5_pdf.pdf?__blob=publicationFile}{EUMETSAT})}
				\label{fig_diagramecode}
			\end{figure}
		\end{minipage}
	\end{minipage}
	\hfill
	\begin{minipage}[c][\textheight][t]{0.48\textwidth}
		\centering
		\includegraphics[width=0.8\textwidth]{img/sarah_show/SISdm20240101}\\
		\textbf{CM-SAF SARAH-3 representation of DNI at SOOI area }\\
		\includegraphics[width=0.8\textwidth]{img/sarah_show/DNIdm20240101}\\
		\textbf{CM-SAF SARAH-3 representation of DNI at SOOI area }
	\end{minipage}
\end{frame}

\begin{frame}
	\frametitle{Comparaison des méthodes de moyennes}
	\begin{figure}[t!]
		\centering
		\includegraphics[width=\linewidth]{img/comparison_mean_method}
		\caption{Comparison between mean used by CM-SAF and mean method used by this study (urmoufia).}
		\label{fig_diffente_mean_method}
	\end{figure}
	\begin{minipage}[c][0.48\textheight][t]{\textwidth}
		\begin{blockbleu}{Mean comparison}
			\textbullet \  pandas mean : \texttt{resample('D').mean()} \\
			\textbullet  \ CM-SAF mean :
			\begin{equation}
				SIS_{DA} = SIS_{CLSDA} \frac{\sum\limits_{i=1}^{n}SIS_i}{\sum\limits_{i=1}^{n}SIS_{CLS_i}}
				\label{SIS_average}
			\end{equation}
			Avec $SIS_{DA}$ la moyenne journalière du SIS,  $SIS_{CLSDA}$ la moyenne journalière du SIS en ciel clair, $SIS_i$ valeur instantanée du SIS par les images satellites et $SIS_{CLS_i}$ est le SIS calculé correspondant au ciel clair.
		\end{blockbleu}
	\end{minipage}
\end{frame}

\subsection{Comparaison des données}

\begin{frame}
	\frametitle{\'Etapes pour la comparaison}
	\begin{columns}[b] % [T] ensures correct vertical alignment
		\begin{column}{0.33\linewidth} % Left column
			\centering
			\includegraphics[width=\textwidth]{img/adapted_station_REUNION}\\
			\textbf{Adaptation} des données géographiques
		\end{column}
		\begin{column}{0.33\linewidth} % Center column
			\centering
			\includegraphics[width=\linewidth]{img/effect_polyorder.png}\\
			\includegraphics[width=\linewidth]{img/effect_window_length.png}\\
			Application du \textbf{filtre Savitzky-Golay}
		\end{column}
		\begin{column}{0.33\linewidth} % Right column
			\centering
			\begin{table}[!h]
				\centering
				\renewcommand{\arraystretch}{1.2} % Ajustement de la hauteur des lignes
				\footnotesize
				\small
				\rowcolors{1}{gray!25}{white}
				\begin{tabular}{>{\footnotesize\bfseries}l | >{\footnotesize}c >{\footnotesize}c}
					\toprule
					\textbf{Daily type}  &\textbf{4 classes} &\textbf{5 classes}\\
					\midrule
					
					Clear sky & \tikz[baseline=-0.5ex]\node[draw,minimum width=1em,minimum height=1em] (box){$\checkmark$}; & \tikz[baseline=-0.5ex]\node[draw,minimum width=1em,minimum height=1em] (box){$\checkmark$}; \\
					Overcast & \tikz[baseline=-0.5ex]\node[draw,minimum width=1em,minimum height=1em] (box){$\checkmark$}; & \tikz[baseline=-0.5ex]\node[draw,minimum width=1em,minimum height=1em] (box){$\checkmark$}; \\
					AM clear & \tikz[baseline=-0.5ex]\node[draw,minimum width=1em,minimum height=1em] (box){$\checkmark$}; & \tikz[baseline=-0.5ex]\node[draw,minimum width=1em,minimum height=1em] (box){$\checkmark$}; \\
					PM clear & \tikz[baseline=-0.5ex]\node[draw,minimum width=1em,minimum height=1em] (box){$\checkmark$}; & \tikz[baseline=-0.5ex]\node[draw,minimum width=1em,minimum height=1em] (box){$\checkmark$}; \\
					Random & \tikz[baseline=-0.5ex]\node[draw,minimum width=1em,minimum height=1em] (box){}; & \tikz[baseline=-0.5ex]\node[draw,minimum width=1em,minimum height=1em] (box){$\checkmark$}; \\
					
					\bottomrule
				\end{tabular}
			\end{table}
			\textbf{Classification} des cycles journaliers
			
		\end{column}
	\end{columns}
\end{frame}




\section{Résultats}
\subsection{Comparaison DNI}

\begin{frame}
	\frametitle{DNI in situ comparée au DNI estimé}
	\begin{figure}[c]
		\centering
		\begin{minipage}[b]{0.49\textwidth}
			\centering
			\includegraphics[width=\linewidth]{img/comparison_curve_of_CHP1_dni_ground_BSRN.png}
			\caption{\centering Comparaison moyenne de DNI au sol (CHP1) avec DNI estimé (CMP22) aux stations BSRN à la Réunion.}
			\label{comparison_curve_of_CHP1_dni_ground_BSRN}
		\end{minipage}\hfill
		\begin{minipage}[b]{0.49\textwidth}
			\centering
			\includegraphics[width=\linewidth]{img/comparison_XY_of_CHP1_dni_ground_BSRN.png}
			\caption{\centering Nuage de points de DNI au sol (CHP1) avec  DNI estimé (CMP22) aux stations BSRN à la Réunion.}
			\label{comparison_XY_of_CHP1_dni_ground_BSRN}
		\end{minipage}
	\end{figure}
\end{frame}



\begin{frame}
	\frametitle{Résultats du site du BSRN}
	\begin{columns}[T] % [T] ensures correct vertical alignment
		\begin{column}{0.49\linewidth} % Left column
			\centering
			\includegraphics[width=\linewidth]{img/comparison_curve_of_CHP1_dni_ground_URBSRN.png}
		\end{column}
		\begin{column}{0.49\linewidth} % Center column
			\centering
			\includegraphics[width=\linewidth]{img/comparison_XY_of_CHP1_dni_ground_URBSRN.png}
		\end{column}
	\end{columns}
\end{frame}


\subsection{Comparaison GHI}
\begin{frame}
	\frametitle{Résultats du site du BSRN}
	\begin{columns}[T] % [T] ensures correct vertical alignment
		\begin{column}{0.49\linewidth} % Left column
			\centering
			\includegraphics[width=\linewidth]{img/comparison_curve_of_CMP22_ghi_URBSRN.png}
		\end{column}
		\begin{column}{0.49\linewidth} % Center column
			\centering
			\includegraphics[width=\linewidth]{img/comparison_XY_of_CMP22_ghi_URBSRN.png}
		\end{column}
	\end{columns}
\end{frame}

\subsection{Indicateur de précision}
\begin{frame}
	\frametitle{Résultats pour toutes les stations (daily/monthly/yearly)}
	\begin{columns}[T] % [T] ensures correct vertical alignment
		\begin{column}{0.49\linewidth} % Left column
			\centering
			\includegraphics[width=0.8\textwidth]{img/stat_indicator_of_DNI}\\
			\includegraphics[width=0.8\textwidth]{img/stat_indicator_of_DNI_monthly}\\
			\includegraphics[width=0.8\textwidth]{img/stat_indicator_of_DNI_yearly}
		\end{column}
		\begin{column}{0.49\linewidth} % Center column
			\centering
			\includegraphics[width=0.8\textwidth]{img/stat_indicator_of_GHI}\\
			\includegraphics[width=0.8\textwidth]{img/stat_indicator_of_GHI_monthly}\\
			\includegraphics[width=0.8\textwidth]{img/stat_indicator_of_GHI_yearly}
		\end{column}
	\end{columns}
\end{frame}

\section{Conclusion et Discussion}
\begin{frame}
	\frametitle{Conclusion et Discussion}
	\begin{columns}[T] % [T] ensures correct vertical alignment
		\begin{column}{0.32\linewidth} % Left column
			\centering
			\begin{blockorange}{Discussions}
				\textbullet \    Non prise en compte de \textbf{données moyennées par CM-SAF}
				
				\textbullet \   Résolution spatiale de \textbf{SARAH-3}
				
				\textbullet \     
			\end{blockorange}
		\end{column}
		\begin{column}{0.32\linewidth} % Center column
			\centering
			\begin{blockbleu}{Conclusion}
				\textbullet \  Le \textbf{GHI} est mieux estimé que le DNI.
				
				\textbullet \  \textbf{92\%} des données de DNI sont sous-estimées.
				
				\textbullet \  \textbf{40\%} des données de GHI sont surestimées .
				
				\textbullet \  Il en ressort que \textbf{SARAH-3} est un outil intéressant pour l'estimation de la ressource solaire dans la zone SOOI. 
			\end{blockbleu}
		\end{column}
		\begin{column}{0.32\linewidth} % Center column
			\centering
			\begin{blockorange}{Perceptives}
				\textbullet \  Appliquer les filtres et les classifications pour les 38 autres stations afin d'\textbf{améliorer les estimations} de SARAH-3.
				
				\textbullet \  Réussir une \textbf{descente d'échelle} pour la résolution du satellite SARAH-3
				
				\textbullet \     
			\end{blockorange}
		\end{column}
	\end{columns}
	\begin{center}
		\includegraphics[width=0.4\textheight]{img/logo_UFRST}
		\quad\quad\quad
		\includegraphics[width=0.4\textheight]{img/energy_labb}
	\end{center}
\end{frame}
	
\end{document}
